\documentclass{article}
\usepackage{utils}

\usepackage{physics}
\usepackage{amsthm,amsfonts,amsmath}

\theoremstyle{definition}
\newtheorem{example}{Пример}


\title{Quantum Machine Learning}
\date{}

\begin{document}

\maketitle

\section{Введение}
В данном документе содержит введение в квантовые вычисления, основные алгоритмы и их применение в машинном обучении.


\subsection{Квантовый компьютер}

Квантовый компьютер --- это вычислительное устройство, которое использует некоторые явления квантовой механики
(суперпозиция состояний (superposition), квантовая запутанность (entanglement)).
Квантовые компьютеры могут решать некоторые задачи значительно быстрее классических архитектур за счет того,
что все операции выполняются сразу над всеми возможными состояниями системы.

\subsection{Кубиты}
В классической архитектуре информация представляется битами, которые могут принимать только одно из двух значений:
$0$ или $1$.
В системе из $L$ битов, соответсвенно, может быть $2^L$ различных состояний.
Как правило, при выполнении операций с системой мы производим операции с одним (может быть несколькими) состоянием системы.

В квантовом же компьютере аналогом бита является квантовый бит --- кубит.
Это некоторая квантовая система, обладающая двумя базисными состояниями ($\ket{0}$ и $\ket{1}$).
Система из $L$ кубитов обладает $2^L$ базисными состояниями.

\begin{example}
При $L = 3$ получаем следующие состояния:
$\ket{000}$, $\ket{001}$, $\ket{010}$, $\ket{011}$, $\ket{100}$, $\ket{101}$, $\ket{110}$, $\ket{111}$.
\end{example}

Для удобства в дальнейшем состояния системы будем обозначать $\ket{j}, j=\overline{1, N}, N = 2^L$.
Общее состояние системы описывается {\em суперпозицией} ее базисных состояний
\[
    \ket{\psi} = \sum_{j=1}^{N} \lambda_j \ket{j}, \quad \sum_{j=1}^N |\lambda_j|^2 = 1,
\]
где $\lambda_j \in \mathbb{C}$ --- это комплексные амплитуды.

\subsection{Базовые операции}

Пространством состояний квантовой системы является $2^L$-мерное гильбертово пространство
(или $N$-мерное в наших обозначениях).
Базовые состояния $\{\ket{j}\}_{j=1}^N$ образуют ортонормироанный базис.
Обозначение $\ket{v}$ соответствует вектор-столбцу $v$, а $\bra{v}$ --- его эрмитово сопряжению $v^\dag$
\[
    v = \begin{pmatrix}
    a_1 \\
    \cdots \\
    a_N 
    \end{pmatrix},
    \quad
    v^\dag = \begin{pmatrix} \overline{a_1} & \cdots & \overline{a_N} \end{pmatrix}.
\]
Тогда скалярное произведение векторов $v, u$ может быть записано как $\bra{v}\ket{u}$.

Над системой можно проводить два типа операций:
\begin{enumerate}
    \item Измерение.
    \item Унитарное преобразование.
\end{enumerate}

\paragraph*{Измерение.}
Едиснтвенный способ получить информацию о состоянии системы это провести операцию {\em измерение}.
Измерение возвращает случайную величину, которая принимает одно из значений $\ket{j}, j=\overline{1, N}$
с вероятностью $|\lambda_j|^2$.
Эта операция необратима, то есть повторное измерение вернет то же самое состояние системы.
Поэтому в квантовых системах нет условных операций \texttt{if ... else ...} (хотя есть некоторые условные операции, например, вентилем controlled-NOT или CNOT, подробнее смотри в следующих разделах).
Спонтанные измерения системы вносят неустойчивость и создают трудности в создании систем с большим количеством кубитов.

\paragraph*{Унитарные преобразования.}


\end{document}
